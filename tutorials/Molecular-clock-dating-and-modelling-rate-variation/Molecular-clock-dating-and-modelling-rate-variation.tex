% This file was created (at least in part) by the script ParseMdtoLatex by Louis du Plessis
% (Available from https://github.com/taming-the-beast)

\documentclass[11pt]{article}
\input{preamble}

% Add your bibtex library here
\addbibresource{master-refs.bib}


%%%%%%%%%%%%%%%%%%%%
% Do NOT edit this %
%%%%%%%%%%%%%%%%%%%%
\begin{document}
\renewcommand{\headrulewidth}{0.5pt}
\headsep = 20pt
\lhead{ }
\rhead{\textsc {BEAST v2 Tutorial}}
\thispagestyle{plain}


%%%%%%%%%%%%%%%%%%
% Tutorial title %
%%%%%%%%%%%%%%%%%%
\begin{center}

	% Enter the name of your tutorial here
	\textbf{\LARGE Tutorial using BEAST v2.4.2}\\\vspace{2mm}

	% Enter a short description of your tutorial here
	\textbf{\textcolor{mycol}{\Large Tutorial template}}\\

	\vspace{4mm}

	% Enter the names of all the authors here
	{\Large {\em Author I. Name}}
\end{center}

Template for an empty tutorial

%%%%%%%%%%%%%%%%%
% Tutorial body %
%%%%%%%%%%%%%%%%%

\section{Background}\label{background}

This is a template tutorial and style guide to help formatting Markdown
tutorials.

Please start the tutorial by adding some background about the tutorial
in this section, clearly explaining the question/problem and the type of
analysis that the methods in the tutorial should be used for. In the
next section please add a short description of all the programs or
packages used in the tutorial. The tutorial exercise should follow this
part. Please add a short explanation on the dataset used in the tutorial
before starting with the exercise. Please also add a section after the
exercise interpreting the results. End your tutorial with some useful
links.

Some of the text in this tutorial template is just dummy filler text.
Please do not try to understand it.

\clearpage

\section{Programs used in this
Exercise}\label{programs-used-in-this-exercise}

\subsubsection{BEAST2 - Bayesian Evolutionary Analysis Sampling Trees
2}\label{beast2---bayesian-evolutionary-analysis-sampling-trees-2}

BEAST2 is a free software package for Bayesian evolutionary analysis of
molecular sequences using MCMC and strictly oriented toward inference
using rooted, time-measured phylogenetic trees \citep{Bouckaert2014}.
This tutorial uses the BEAST2 version 2.4.2.

\clearpage

\section{Practical: Exercise title}\label{practical-exercise-title}

Lorem ipsum dolor sit amet, consectetur adipiscing elit. Fusce porta
augue id vulputate iaculis. Sed non posuere lectus. Integer at magna
quis nulla tempus cursus. Integer ut nisl elit. Nam pellentesque
pharetra orci eu facilisis. Nullam vitae leo tempus nunc consequat
finibus ut ut nisi. Phasellus vitae faucibus dolor, id venenatis lacus.
Sed eu lacus a nibh luctus semper. Proin non tellus odio. Duis elementum
lorem eget nisl rhoncus feugiat. In hendrerit vehicula purus. Aliquam
ornare libero quis tincidunt efficitur. Nam sapien augue, mattis nec
hendrerit ut, commodo in nisi.

\subsection{This is a subsection}\label{this-is-a-subsection}

Quisque a dictum erat. Curabitur congue sapien sit amet pharetra
pretium. Proin posuere euismod velit, eget faucibus ex varius id. Fusce
sodales maximus malesuada. Mauris auctor dui in justo interdum egestas.
Cras dapibus commodo nulla vitae congue. Vestibulum sit amet justo sit
amet ex pretium bibendum. Donec ac mollis lorem, vel semper enim.
Suspendisse sit amet auctor dui. Nullam ac efficitur mauris. Proin
aliquam tincidunt felis nec semper. Vestibulum vestibulum, eros sit amet
consectetur blandit, elit dolor posuere sem, a porta purus odio sit amet
quam. Quisque dapibus erat sem, at vulputate libero dapibus sit amet.
Mauris rhoncus odio nisl, nec interdum lacus consequat nec. Class aptent
taciti sociosqu ad litora torquent per conubia nostra, per inceptos
himenaeos.

Etiam tincidunt porttitor rutrum. Nulla facilisi. Mauris vehicula, justo
ac ultricies tempus, quam erat hendrerit dui, vel pharetra sapien nibh
vel ex. Sed molestie eu dui in laoreet. Pellentesque ultrices, orci
vitae lacinia suscipit, erat sapien elementum ligula, sit amet viverra
ante lorem eget elit. Cras euismod felis libero, pharetra lobortis arcu
congue vehicula. Nullam posuere dapibus mauris, eget vulputate ligula
auctor eget. Aenean et tempus est. Aliquam vehicula arcu vitae metus
dictum viverra. Aliquam vitae purus mauris. Nullam interdum mauris eget
sagittis consequat. Quisque in orci elementum, eleifend tortor eget,
bibendum orci. Etiam aliquet dolor non neque semper fermentum. Praesent
vitae venenatis mi, ut faucibus ligula. Phasellus vitae lorem neque.
Interdum et malesuada fames ac ante ipsum primis in faucibus.

\subsubsection{This is a sub-subsection}\label{this-is-a-sub-subsection}

Etiam posuere urna ut condimentum sagittis. Suspendisse posuere, ex nec
eleifend fringilla, nisl augue posuere augue, elementum mollis justo
felis sed purus. Cum sociis natoque penatibus et magnis dis parturient
montes, nascetur ridiculus mus. Mauris efficitur eros ut turpis
elementum vestibulum. Sed sit amet nisi at nunc luctus laoreet id ac
enim. Aliquam elementum risus id urna dictum fringilla. Aenean lobortis,
risus euismod molestie pulvinar, massa odio pharetra nulla, vitae
facilisis neque magna sed lorem. Praesent ipsum enim, commodo ut
pharetra in, sollicitudin ac massa. Donec et interdum mauris. Ut
molestie, risus quis fermentum placerat, diam risus posuere nisi, eget
viverra tortor neque ac sem. Donec viverra magna non dolor aliquam, in
suscipit massa facilisis. Suspendisse congue arcu sed risus consectetur
commodo. Aenean metus odio, volutpat at tincidunt id, ullamcorper in
dui. Cum sociis natoque penatibus et magnis dis parturient montes,
nascetur ridiculus mus. Cras ut sem in odio sodales iaculis non quis
neque.

Quisque non mollis massa, nec eleifend dolor. Proin porta elit metus, a
lobortis enim venenatis ac. Nam scelerisque consectetur mi et gravida.
Vestibulum placerat, est vitae euismod finibus, purus nisl viverra quam,
eget condimentum mauris magna vel nisl. Phasellus pretium vitae diam in
volutpat. Cras gravida non quam ut consectetur. Vivamus congue vulputate
lorem.

\subsection{This is another
subsection}\label{this-is-another-subsection}

Praesent sodales est in tempor commodo. Suspendisse nulla metus, gravida
eget malesuada vel, viverra eu felis. In vitae leo facilisis, ornare
nunc nec, tempor tortor. Duis pretium mi eros, at consequat neque
tincidunt eget. Mauris vestibulum venenatis arcu, eget lacinia arcu
faucibus ut. Phasellus aliquam dui ipsum, a eleifend lacus fermentum at.
Suspendisse congue orci quis ante consequat ornare. Integer a massa
blandit, vestibulum eros ut, pulvinar augue. Class aptent taciti
sociosqu ad litora torquent per conubia nostra, per inceptos himenaeos.

Quisque a urna a massa congue rhoncus. Donec bibendum tempus velit. Nam
varius augue sit amet lacinia hendrerit. Proin tincidunt massa ut mi
vestibulum placerat. Phasellus eget dui molestie, aliquet libero
efficitur, vehicula ex. Pellentesque ultricies ante leo, eu lobortis
odio convallis id. Donec vitae risus dui. Nulla orci velit, ultricies
sed finibus quis, blandit quis arcu. Morbi non neque non odio rutrum
condimentum. Vivamus libero metus, vehicula vitae elit ac, tincidunt
pretium dui. Proin condimentum fringilla diam, blandit blandit nisl
dapibus vel. Proin ante felis, accumsan eget ligula et, lobortis dictum
nunc. Mauris a ante dignissim ipsum tincidunt tristique.

\clearpage

\section{Tutorial style guide}\label{tutorial-style-guide}

\subsection{Text styling}\label{text-styling}

This is how to write \emph{italic text}.

This is how to write \textbf{bold text}.

This is how to write \textbf{\emph{bold and italic text}}.

Do text superscripts like this 7$^{th}$, x$^{2y}$ or x$^{2y + 3z}$.

\subsection{Lists}\label{lists}

\subsubsection{Unnumbered lists}\label{unnumbered-lists}

\begin{itemize}

\item
  Lorem ipsum dolor sit amet, consectetur adipiscing elit.
\item
  Integer pharetra arcu ut nisl mollis ultricies.

  \begin{itemize}
  
  \item
    Fusce nec tortor at enim cursus dictum.
  \item
    Phasellus nec urna quis velit eleifend convallis sodales nec augue.
  \end{itemize}
\item
  In iaculis turpis in massa facilisis, quis ultricies nibh ultricies.
\item
  Nam vitae turpis eu lacus imperdiet mollis id at augue.
\item
  Sed sed turpis ac dolor mollis accumsan.
\end{itemize}

\subsubsection{Numbered lists}\label{numbered-lists}

\begin{enumerate}
\def\labelenumi{\arabic{enumi}.}

\item
  Lorem ipsum dolor sit amet, consectetur adipiscing elit.
\item
  Integer pharetra arcu ut nisl mollis ultricies.

  \begin{enumerate}
  \def\labelenumii{\arabic{enumii}.}
  
  \item
    Fusce nec tortor at enim cursus dictum.
  \item
    Phasellus nec urna quis velit eleifend convallis sodales nec augue.
  \end{enumerate}
\item
  In iaculis turpis in massa facilisis, quis ultricies nibh ultricies.
\item
  Nam vitae turpis eu lacus imperdiet mollis id at augue.
\item
  Sed sed turpis ac dolor mollis accumsan.
\end{enumerate}

\subsubsection{Mixed lists}\label{mixed-lists}

\begin{enumerate}
\def\labelenumi{\arabic{enumi}.}

\item
  Lorem ipsum dolor sit amet, consectetur adipiscing elit.
\item
  Integer pharetra arcu ut nisl mollis ultricies.

  \begin{itemize}
  
  \item
    Fusce nec tortor at enim cursus dictum.
  \item
    Phasellus nec urna quis velit eleifend convallis sodales nec augue.
  \end{itemize}
\item
  In iaculis turpis in massa facilisis, quis ultricies nibh ultricies.
\item
  Nam vitae turpis eu lacus imperdiet mollis id at augue.
\item
  Sed sed turpis ac dolor mollis accumsan.
\end{enumerate}

\subsection{Figures}\label{figures}

\begin{figure}
    \centering
    \includegraphics[width=0.250000\textwidth]{figures/Logo_bw.png}
    \caption{This figure is 25\% of the page width.}
    \label{fig:example1}
\end{figure}

\begin{figure}
    \centering
    \includegraphics[width=0.100000\textwidth]{figures/Logo_bw.png}
    \caption{This figure is only 10\% of the page width.}
    \label{fig:example2}
\end{figure}

\section{Code}\label{code}

A bit of inline monospaced font can be made \lstinline!like this!.
Larger code blocks can be made by using the code environment:

Java:

\begin{lstlisting}[language=Java]
public class HelloWorld {

    public static void main(String[] args) {
        // Prints "Hello, World" to the terminal window.
        System.out.println("Hello, World");
    }

}
\end{lstlisting}

XML:

\begin{lstlisting}[language=XML]
    <BirthDeathSkylineModel spec="BirthDeathSkylineModel" id="birthDeath" tree="@tree" contemp="true">
          <parameter name="origin" id="origin" value ="100" lower="0."/>    
          <parameter name="R0" id="R0" value="2" lower="0." dimension ="10"/>
          <parameter name="becomeUninfectiousRate" id="becomeUninfectiousRate" value="1" lower="0." dimension ="10"/>
          <parameter name="samplingProportion" id="samplingProportion" value="0."/>
          <parameter name="rho" id="rho" value="1e-6" lower="0." upper="1."/>
    </BirthDeathSkylineModel>
\end{lstlisting}

R:

\begin{lstlisting}[language=R]
    > myString <- "Hello, World!"
    > print (myString)
    [1] "Hello, World!"
\end{lstlisting}

\section{Equations}\label{equations}

Inline equations: $ \dot{x} = \sigma(y-x) $

Displayed equations:

\begin{equation}
    \left( \sum_{k=1}^n a_k b_k \right)^2 \leq \left( \sum_{k=1}^n a_k^2 \right) \left( \sum_{k=1}^n b_k^2 \right)
\end{equation}

\subsection{Instruction boxes}\label{instruction-boxes}

Use block-quotes for step-by-step instruction that the user should
perform (this will produce a framed box on the website):

\begin{framed}
The data we have is not the data we want, and the data we need is not
the data we have.

We can input \textbf{any} formatted text in here:

\begin{itemize}

\item
  Even
\item
  Lists
\end{itemize}

or equations:

\begin{equation}
(x_1, \ldots, x_n) \left( \begin{array}{ccc}
\phi(e_1, e_1) & \cdots & \phi(e_1, e_n) \\
\vdots & \ddots & \vdots \\
\phi(e_n, e_1) & \cdots & \phi(e_n, e_n)
\end{array} \right)
\left( \begin{array}{c}
y_1 \\
\vdots \\
y_n
\end{array} \right)
\end{equation}
\end{framed}

\section{Hyperlinks}\label{hyperlinks}

Add links to figures like this:

\begin{itemize}

\item
  Figure \ref{fig:example1} is 25\% of the page width.
\item
  Figure \ref{fig:example2} is 10\% of the page width.
\end{itemize}

Add links to external URLs like \href{http://www.google.com}{this}.

Links to equations or different sections within the same document are a
little buggy.

\clearpage

\section{Useful Links}\label{useful-links}

\begin{itemize}

\item
  \href{http://www.beast2.org/book.html}{Bayesian Evolutionary Analysis
  with BEAST 2} \citep{BEAST2book2014}
\item
  BEAST 2 website and documentation: \url{http://www.beast2.org/}
\item
  BEAST 1 website and documentation: \url{http://beast.bio.ed.ac.uk}
\item
  Join the BEAST user discussion:
  \url{http://groups.google.com/group/beast-users}
\end{itemize}

\clearpage



%%%%%%%%%%%%%%%%%%%%%%%
% Tutorial disclaimer %
%%%%%%%%%%%%%%%%%%%%%%%
% Please do not change the license
% Add the author names and relevant links
% Add any other aknowledgments here
\href{http://creativecommons.org/licenses/by/4.0/}{\includegraphics[scale=0.8]{figures/ccby.pdf}} This tutorial was written by Author I. Name for \href{https://taming-the-beast.github.io}{Taming the BEAST} and is licensed under a \href{http://creativecommons.org/licenses/by/4.0/}{Creative Commons Attribution 4.0 International License}. 


%%%%%%%%%%%%%%%%%%%%
% Do NOT edit this %
%%%%%%%%%%%%%%%%%%%%
Version dated: \today



\newpage

%%%%%%%%%%%%%%%%
%  REFERENCES  %
%%%%%%%%%%%%%%%%

\printbibliography[heading=relevref]


\end{document}