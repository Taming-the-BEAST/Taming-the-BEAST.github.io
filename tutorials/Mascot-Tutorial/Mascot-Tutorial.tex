% This file was created (at least in part) by the script ParseMdtoLatex by Louis du Plessis
% (Available from https://github.com/taming-the-beast)

\documentclass[11pt]{article}
\input{preamble}

% Add your bibtex library here
\addbibresource{master-refs}


%%%%%%%%%%%%%%%%%%%%
% Do NOT edit this %
%%%%%%%%%%%%%%%%%%%%
\begin{document}
\renewcommand{\headrulewidth}{0.5pt}
\headsep = 20pt
\lhead{ }
\rhead{\textsc {BEAST v2 Tutorial}}
\thispagestyle{plain}


%%%%%%%%%%%%%%%%%%
% Tutorial title %
%%%%%%%%%%%%%%%%%%
\begin{center}

	% Enter the name of your tutorial here
	\textbf{\LARGE Tutorial using BEAST v2.4.7}\\\vspace{2mm}

	% Enter a short description of your tutorial here
	\textbf{\textcolor{mycol}{\Large MASCOT Tutorial}}\\

	\vspace{4mm}

	% Enter the names of all the authors here
	{\Large {\em Nicola F. Müller}}
\end{center}

Parameter and State inference using the approximate structured
coalescent

%%%%%%%%%%%%%%%%%
% Tutorial body %
%%%%%%%%%%%%%%%%%

\section{Background}\label{background}

Phylogeographic methods can help reveal the movement of genes between
populations of organisms. This has been widely used to quantify pathogen
movement between different host populations, the migration history of
humans, and the geographic spread of languages or the gene flow between
species using the location or state of samples alongside sequence data.
Phylogenies therefore offer insights into migration processes not
available from classic epidemiological or occurrence data alone.

The structured coalescent allows to coherently model the migration and
coalescent process, but struggles with complex datasets due to the need
to infer ancestral migration histories. Thus, approximations to the
structured coalescent, which integrate over all ancestral migration
histories, have been developed. This tutorial gives an introduction into
how a MASCOT analysis in BEAST2 can be set-up. MASCOT is short for
\textbf{M}arginal \textbf{A}pproximation of the \textbf{S}tructured
\textbf{CO}alescen\textbf{T} and implements a structured coalescent
approximation \citep{Mueller2017}. This approximation doesn't require
migration histories to be sampled using MCMC and therefore allows to
analyse phylogenies with more than three or four states. \clearpage

\section{Programs used in this
Exercise}\label{programs-used-in-this-exercise}

\subsubsection{BEAST2 - Bayesian Evolutionary Analysis Sampling Trees
2}\label{beast2---bayesian-evolutionary-analysis-sampling-trees-2}

BEAST2 (\url{http://www.beast2.org}) is a free software package for
Bayesian evolutionary analysis of molecular sequences using MCMC and
strictly oriented toward inference using rooted, time-measured
phylogenetic trees. This tutorial is written for BEAST v2.4.7 \citep{BEAST2book2014}.

\subsubsection{BEAUti2 - Bayesian Evolutionary Analysis
Utility}\label{beauti2---bayesian-evolutionary-analysis-utility}

BEAUti2 is a graphical user interface tool for generating BEAST2 XML
configuration files.

Both BEAST2 and BEAUti2 are Java programs, which means that the exact
same code runs on all platforms. For us it simply means that the
interface will be the same on all platforms. The screenshots used in
this tutorial are taken on a Mac OS X computer; however, both programs
will have the same layout and functionality on both Windows and Linux.
BEAUti2 is provided as a part of the BEAST2 package so you do not need
to install it separately.

\subsubsection{TreeAnnotator}\label{treeannotator}

TreeAnnotator is used to summarise the posterior sample of trees to
produce a maximum clade credibility tree. It can also be used to
summarise and visualise the posterior estimates of other tree parameters
(e.g.~node height).

TreeAnnotator is provided as a part of the BEAST2 package so you do not
need to install it separately.

\subsubsection{Tracer}\label{tracer}

Tracer (\url{http://tree.bio.ed.ac.uk/software/tracer}) is used to
summarise the posterior estimates of the various parameters sampled by
the Markov Chain. This program can be used for visual inspection and to
assess convergence. It helps to quickly view median estimates and 95\%
highest posterior density intervals of the parameters, and calculates
the effective sample sizes (ESS) of parameters. It can also be used to
investigate potential parameter correlations. We will be using Tracer
v1.6.0.

\subsubsection{FigTree}\label{figtree}

FigTree (\url{http://tree.bio.ed.ac.uk/software/figtree}) is a program
for viewing trees and producing publication-quality figures. It can
interpret the node-annotations created on the summary trees by
TreeAnnotator, allowing the user to display node-based statistics
(e.g.~posterior probabilities). We will be using FigTree v1.4.2. \clearpage

\section{Practical: Parameter and State inference using the approximate
structured
coalescent}\label{practical-parameter-and-state-inference-using-the-approximate-structured-coalescent}

In this tutorial we will estimate migration rates, effective population
sizes and locations of internal nodes using the marginal approximation
of the structured coalescent implemented in BEAST2, MASCOT.

The aim is to:

\begin{itemize}

\item
  Learn how to infer structure from trees with sampling location
\item
  Get to know how to choose the set-up of such an analysis
\item
  Learn how to read the output of a MASCOT analysis
\end{itemize}

\subsection{Setting up an analysis in
BEAUti}\label{setting-up-an-analysis-in-beauti}

\subsubsection{Download MASCOT}\label{download-mascot}

First, we have to download the package MASCOT using the BEAUTi package
manager. Go to \emph{File \textgreater{}\textgreater{} Manage Packages}
and download the package MASCOT.

\begin{figure}
    \centering
    \includegraphics[width=0.500000\textwidth]{figures/MascotDownload.png}
    \caption{Download the MASCOT package.}
    \label{fig:example1}
\end{figure}

MASCOT will only be available in BEAUti once you close and restart the
program.

\subsubsection{Loading the template}\label{loading-the-template}

Next, we have to load the BEAUTi template from \emph{File}, select
\emph{Template \textgreater{}\textgreater{} Mascot}.

\subsubsection{Loading the Influenza A/H3N2 Sequences
(Partitions)}\label{loading-the-influenza-ah3n2-sequences-partitions}

The sequences from the \emph{data} folder name \emph{H3N2.nexus} can be
either drag and dropped into BEAUti or added by going to \emph{File
\textgreater{}\textgreater{} Import Alignment}. Once the sequences are
added, we need to specify the sampling dates and locations.

\subsubsection{Get the sampling times (Tip
Dates)}\label{get-the-sampling-times-tip-dates}

After clicking the \emph{Auto-configure} button, the sampling times can
be guessed. The sampling times are encoded in the sequence names and are
in the third group after splitting on the vertical bar ``\textbar{}''.
The first group after splitting is the name of the sequence, the second
group contains the accession numbers. The third are the sampling times
and the fourth are the sampling location.

\begin{figure}
    \centering
    \includegraphics[width=0.700000\textwidth]{figures/TipDates.png}
    \caption{Guess sampling times.}
    \label{fig:example1}
\end{figure}

After guessing the sampling times, the column \textbf{Date} should now
have values between 2000 and 2002 and the column \textbf{Height} should
have values from 0 to 2. The heights denote the time difference from a
sequence to the most recently sampled sequence. If everything is
specified correctly, the sequence with Height 0.0 should have Date
2001.9. Next, the sampling locations need to be specified.

\subsubsection{Get the sampling locations (Tip
Locations)}\label{get-the-sampling-locations-tip-locations}

As for the sampling times, they can be guessed from the sequence names.
Initially the column \textbf{Location} should be NOT\_SET for every
sequence. After clicking the \emph{Guess} button, you can split the
sequence on the vertical bar ``\textbar{}'' again. As said before, the
locations are in the fourth group. After clicking the \emph{OK} button,
the window should now look like in the figure below:

\begin{figure}
    \centering
    \includegraphics[width=0.700000\textwidth]{figures/TipLocations.png}
    \caption{Guess sampling locations.}
    \label{fig:example1}
\end{figure}

\subsubsection{Specify the Site Model (Site
Model)}\label{specify-the-site-model-site-model}

Next, we have to specify the site model. For Influenza Hemagluttanin
sequences as we have here, HKY is the most commonly used model of
nucleotide evolution. It allows for difference in transversion and
transition rates. Meaning that changes between bases that are chemically
closer related (transitions) are allowed to have a different rate than
changes between bases that chemically more distinct (transversion).
Additionally, we should allow for different rate categories for
different sires in the alignment. This can be done by setting the
\emph{Gamma Category Count} to 4, which is just a value that has
typically been used. Make sure that estimate is checked next to the
shape parameter. To reduce the number of parameters we have to estimate,
we can set Frequencies to Empirical.

\begin{figure}
    \centering
    \includegraphics[width=0.700000\textwidth]{figures/SiteModel.png}
    \caption{Set the site model.}
    \label{fig:example1}
\end{figure}

\subsubsection{Set the clock model (Clock
Model)}\label{set-the-clock-model-clock-model}

For rapidly evolving viruses, the assumption of a strict molecular clock
is often made, meaning that the molecular clock is the same on each
branch of the phylogeny. To decrease the burnin phase, we can set the
initial value to 0.005.

\begin{figure}
    \centering
    \includegraphics[width=0.700000\textwidth]{figures/ClockRate.png}
    \caption{Set the initial clock rate.}
    \label{fig:example1}
\end{figure}

\subsubsection{Specify the priors and set dimensions
(Priors)}\label{specify-the-priors-and-set-dimensions-priors}

Now, we need to set the priors as well as the dimensions of the
effective population sizes and the migration rates. For this example we
have sequences from Hong Kong, New Zealand and, New York . Overall we
have three different locations, meaning that we need an effective
population size for each of these locations. You can set the dimension
of the effective population size by pressing the \emph{initial} button.
A window will then appear where you can set the dimension to 3. Next, we
can change the prior to a Log Normal prior with M=0 and S=1. Since we
have only a few samples per location, meaning little information about
the different effective population sizes, we will need an informative
prior.

\begin{figure}
    \centering
    \includegraphics[width=0.700000\textwidth]{figures/SetNeDimension.png}
    \caption{Set the dimension for the effective population sizes to 3.}
    \label{fig:example1}
\end{figure}

Next, we have to set the dimension of the migration rate parameter. A
lineage from any of the 3 locations can migrate to 2 (3-1) other
locations. Overall, we therefore have to estimate 3*(3-1) migration
rates and have to set the dimension accordingly. The exponential
distribution as a prior on the migration rate puts much weight on lower
values while not prohibiting larger ones. For migration rates, a prior
that prohibits too large values while not greatly distinguishing between
very small and very very small values (such as the inverse uniform) is
generally a good choice.

\begin{figure}
    \centering
    \includegraphics[width=0.700000\textwidth]{figures/SetMigrationDimension.png}
    \caption{Set the dimension of the migration rates to 6.}
    \label{fig:example1}
\end{figure}

Next, we have to set a prior for the clock rate. Since we only have a
narrow time window of less than a year and only 24 sequences, there
isn't much information in the data about the clock rate. We have however
a good idea about it for Influenza A/H3N2 Hemagglutinin. We can
therefore set the prior to be normally distributed around 0.005
substitution per site and year with a variance of 0.0001. (At this point
we could also just fix the rate)

\subsubsection{Specify the MCMC chain length
(MCMC)}\label{specify-the-mcmc-chain-length-mcmc}

Here we can set the length of the MCMC chain and after how many
iterations the parameter and trees a logged. For this dataset, 2 million
iterations should be sufficient. In order to have enough samples but not
create too large files, we can set the logEvery to 2500, so we have 801
samples overall. Next, we have to save the \lstinline!*.xml! file under
\emph{File \textgreater{}\textgreater{} Save as}.

\begin{figure}
    \centering
    \includegraphics[width=0.700000\textwidth]{figures/MCMC.png}
    \caption{save the *.xml.}
    \label{fig:example1}
\end{figure}

\subsubsection{Run the Analysis using
BEAST2}\label{run-the-analysis-using-beast2}

Run the \lstinline!*.xml! using BEAST2 or use finished runs from the
\emph{precooked-runs} folder. The analysis should take about 6 to 7
minutes.

\subsubsection{Analyse the log file using
Tracer}\label{analyse-the-log-file-using-tracer}

First, we can open the \lstinline!*.log! file in tracer to check if the
MCMC has converged. The ESS value should be above 200 for almost all
values and especially for the posterior estimates. The burnin taken by
Tracer is 10\%, but for this analysis 1\% is enough.

\begin{figure}
    \centering
    \includegraphics[width=0.700000\textwidth]{figures/LogPosterior.png}
    \caption{Check if the posterior converged.}
    \label{fig:example1}
\end{figure}

Next, we can have a look at the inferred effective population sizes. New
York is inferred to have the largest effective population size before
Hong Kong and New Zealand. This tells us that two lineages that are in
the New Zealand are expected to coalesce quicker than two lineages in
Hong Kong or New York.

\begin{figure}
    \centering
    \includegraphics[width=0.700000\textwidth]{figures/LogNe.png}
    \caption{Compare the different inferred effective population sizes.}
    \label{fig:example1}
\end{figure}

In this example, we have relatively little information about the
effective population sizes of each location. This can lead to estimates
that are greatly informed by the prior. Additionally, there can be great
differences between median and mean estimates. The median estimates are
generally more reliable since they are less influence by extreme values.

\begin{figure}
    \centering
    \includegraphics[width=0.700000\textwidth]{figures/MeanMedian.png}
    \caption{Differences between Mean and Meadian estimates.}
    \label{fig:example1}
\end{figure}

We can then look at the inferred migration rates. The migration rates
have the label b\_migration.*, meaning that they are backwards in time
migration rates. The highest rates are from New York to Hong Kong.
Because they are backwards in time migration rates, this means that
lineages from New York are inferred to be likely from Hong Kong if we're
going backwards in time. In the inferred phylogenies, we should
therefore make the observation that lineages ancestral to samples from
New York are inferred to be from the Hong Kong backwards.

\begin{figure}
    \centering
    \includegraphics[width=0.700000\textwidth]{figures/LogMigration.png}
    \caption{Compare the inferrred migration rates.}
    \label{fig:example1}
\end{figure}

\subsubsection{Make the MCC tree using
TreeAnnotator}\label{make-the-mcc-tree-using-treeannotator}

Next, we want to summarize the trees. This we can do using
treeAnnotator. Open the programm and then set the options as below. You
have to specify the \emph{Burnin precentage}, the \emph{Node heights},
\emph{Input Tree File} and the \emph{Output File} after clicking
\emph{Run} the programm should summarize the trees.

\begin{figure}
    \centering
    \includegraphics[width=0.500000\textwidth]{figures/TreeAnnotator.png}
    \caption{Make the maximum clade credibility tree.}
    \label{fig:example1}
\end{figure}

\subsubsection{Check the MCC tree using
FigTree}\label{check-the-mcc-tree-using-figtree}

We can now open the MCC tree using FigTree. The output contains several
things. Each node has several traits. Among them are those called
Hong\_Kong, New\_York and New\_Zealand. The value of those traits is the
probability of that node being in that location as inferred using
MASCOT.

\begin{figure}
    \centering
    \includegraphics[width=1.000000\textwidth]{figures/HongKongLabels.png}
    \caption{Compare the inferred node probabilities.}
    \label{fig:example1}
\end{figure}

We can now check if lineages ancestral to samples from New York are
actually inferred to be from Hong Kong, or the probability of the root
being in any of the locations. It should here be mentioned that the
inference of nodes being in a particular location makes some simplifying
assumptions, such as that there are no other locations where lineages
could have been. \clearpage

\section{Useful Links}\label{useful-links}

\begin{itemize}

\item
  MASCOT source code: \url{https://github.com/nicfel/Mascot}
\item
  \href{http://www.beast2.org/book.html}{Bayesian Evolutionary Analysis
  with BEAST 2} \citep{BEAST2book2014}
\item
  BEAST 2 website and documentation: \url{http://www.beast2.org/}
\item
  Join the BEAST user discussion:
  \url{http://groups.google.com/group/beast-users} 
\end{itemize}



%%%%%%%%%%%%%%%%%%%%%%%
% Tutorial disclaimer %
%%%%%%%%%%%%%%%%%%%%%%%
% Please do not change the license
% Add the author names and relevant links
% Add any other aknowledgments here
\href{http://creativecommons.org/licenses/by/4.0/}{\includegraphics[scale=0.8]{figures/ccby.pdf}} This tutorial was written by Nicola F. Müller for \href{https://taming-the-beast.github.io}{Taming the BEAST} and is licensed under a \href{http://creativecommons.org/licenses/by/4.0/}{Creative Commons Attribution 4.0 International License}. 


%%%%%%%%%%%%%%%%%%%%
% Do NOT edit this %
%%%%%%%%%%%%%%%%%%%%
Version dated: \today




%%%%%%%%%%%%%%%%
%  REFERENCES  %
%%%%%%%%%%%%%%%%

\printbibliography[heading=relevref]


\end{document}